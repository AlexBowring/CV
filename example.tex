\documentclass{my_cv}
\usepackage[skins]{tcolorbox}
\usepackage{lipsum}
\usepackage[normalem]{ulem}
\usepackage{scrextend}
\usepackage{enumitem}
\usepackage{fontspec}

\newcommand*{\myfont}{\fontfamily{lmr}\selectfont}
\DeclareTextFontCommand{\textmyfont}{\itshape\myfont}

%Setup hyperref package, and colours for links
\usepackage{hyperref}
\definecolor{linkcolour}{rgb}{0,0.2,0.6}
\hypersetup{colorlinks,breaklinks,urlcolor=linkcolour, linkcolor=linkcolour}

\begin{document}
\begin{multicols}{2}[
    \titletext{Alexander}%
        {Bowring}%
        {Population Health PhD, Mathematics BSc}%
        {(+44) 7944 240731}%
        {\href{mailto:alex.bowring@bdi.ox.ac.uk}{alex.bowring@bdi.ox.ac.uk}}%
        {\href{https://www.linkedin.com/in/alexander-bowring/}{alexander-bowring}}%
        {\href{https://github.com/AlexBowring/}{AlexBowring}}
]

\end{multicols}

\vspace{-1.7cm}
\begin{tcolorbox}[colback=white,colframe=cyan,width=\dimexpr\textwidth+12mm\relax,enlarge left by=-6mm]
\section{\faStickyNote}{Summary}
\textbf{Early Career Research Fellow} and \textbf{Population Health PhD} graduate from the \textbf{University of Oxford} with a passion for \textbf{developing and advancing} methods to analyse \textbf{fMRI data}. As a research scientist I have \textbf{processed and analysed large fMRI datasets}, using the three main fMRI software packages (\textbf{AFNI, FSL, SPM}) and applying programming skills (\textbf{Python, R, Matlab, Unix}) to conduct data manipulation and develop replicable analysis workflows. These skills are combined with a technical background in \textbf{mathematics}, for which I obtained a \textbf{First Class Honours BSc degree} from the \textbf{University of Warwick}. 
  
\end{tcolorbox}


\section{\faGraduationCap}{Education}

\work{Nuffield Department of Population Health, University of Oxford  \hfill Oxford, UK}{\uline{\textmyfont{Doctor of Philosophy in Population Health} \hfill \textmyfont{Oct 2016 - Nov 2019}}}%
{\begin{addmargin}[2em]{1em}
    Thesis: \href{https://doi.org/10.31237/osf.io/mj7qa}{On the Reproducibility and Interpretability of Group-Level Task-fMRI Results} \\
    Supervisors: Professor Thomas Nichols, Professor Stephen Smith \\
    Examiners: Professor Mark Woolrich, Professor Ian Dryden
    \begin{itemize}[topsep=0pt,itemsep=0pt,partopsep=0pt, parsep=0pt] 
    \item Developed a novel `Confidence Sets' method for group-level inference on task-fMRI effect-size maps. Similar to confidence intervals, the Confidence Sets convey information about brains regions where effect sizes have exceeded and fallen short of a \textit{non-zero} threshold. This research led to two publications in the \textit{NeuroImage} journal, where the Confidence Sets were developed for application to raw \%BOLD and Cohen's \textit{d} effect-size images, respectively.  
    \item Carried out a project to assess the reproducibility of task-fMRI results when different software packages are used to analyse the data. Reproduced the results of three published task-fMRI studies in the three main fMRI software packages (AFNI,FSL,SPM), and applied a range of quantitative methods to assess the similarity between the results. This research was published in the \textit{Human Brain Mapping} journal, and has an Attention Score ranking it in the top 5\% of all research outputs scored by Altmetric.  
    \end{itemize}
\end{addmargin}}
    
\work{University of Warwick \hfill Coventry, UK}{\uline{\textmyfont{Batchelor of Science in Mathematics} \hfill \textmyfont{Sep 2012 - Jul 2015}}}%
    {\begin{addmargin}[2em]{1em}
    Graduated with \textbf{First Class Honours}, achieving a score of >80\% in modules including: Algebra II, Analysis III, Galois Theory, Programming for Scientists and Mathematics by Computer.
\end{addmargin}}
    
\section{\faBriefcase}{Professional Experience}

\work{Nuffield Department of Population Health, University of Oxford \hfill Oxford, UK}{\uline{\textmyfont{Early Career Research Fellow in Neuroimaging Statistics} \hfill \textmyfont{Nov 2019 - Present}}}%
    {\begin{itemize}[topsep=0pt,itemsep=0pt,partopsep=0pt, parsep=0pt] 
    \item Led a research project to investigate the variability of MRI results attributable to differences in the numerous analytic strategies that are implemented to process functional imaging data. 
    \item Designed and carried out a `multiverse' of analysis workflows on open MRI datasets, using Python, Matlab and Unix programming to conduct multiple analyses in parallel on a computing cluster.
    \item Developed statistical methods to assess the similarity of spatial statistical maps, and created Python notebooks to apply these methods and visualise the results.
    \item Led the `Neuroimaging Statistics Oxford' reading group, where I organised weekly talks at the university from international expert speakers in the field of neuroimaging statistics. 
    \item Contributed to wider academics activities within the university that included teaching PhD students statistical methods, programming skills, and aspects of performing independent research.
    \end{itemize}
    }%

\work{Warwick Manufacturing Group, University of Warwick \hfill Coventry, UK}{\uline{\textmyfont{Research Assistant in Neuroimaging Statistics} \hfill \textmyfont{Nov 2015 - Oct 2016}}}%
    {\begin{itemize}[topsep=0pt,itemsep=0pt,partopsep=0pt, parsep=0pt] 
    \item Developed standard practices for data-sharing and meta-analysis in neuroimaging as part of the project ``Transforming Statistical Methodology for Neuroimaging Meta-Analysis”. 
    \item Carried out extensive analyses of fMRI data, gaining substantial experience in the main software packages used to analyse imaging data.
    \item Documented my research efforts and provided version control of my analyses with Github. Used online data repositories to share my analysis results. 
    \item Mentored a PhD student that visited our lab for three months. The research project we carried out in this time frame was subsequently published in a peer-reviewed journal. 
    \end{itemize}
    }%
     
\work{Undergraduate Research Support Scheme, University of Warwick \hfill Coventry, UK}%
    {\uline{\textmyfont{Undergraduate Research Intern} \hfill \textmyfont{Jun 2015 - Oct 2015}}}%
    {\begin{itemize}[topsep=0pt,itemsep=0pt,partopsep=0pt, parsep=0pt] 
    \item Assisted in developing software that provided visualisations of neuroimaging results for an undergraduate research project ``Visualising the brain - Developing viewers of standardised fMRI results". 
    \item Conducted analyses of neuroimaging data to test and debug research prototypes of the viewer. 
    \item Gained experience in academic research, attending department meetings alongside PhD students and post-doctoral researchers and communicating daily with my supervisor.
    \end{itemize}
    }%
    
\section{\faBook}{Publications}
\leftskip=2em
\parindent=-2em

\hspace{-2em}\textbf{Bowring A}, Telschow F, Schwartzman A, Nichols TE. Confidence Sets for Cohen's \textmyfont{d} effect size images. \textmyfont{NeuroImage.} 2021. 

\textbf{Bowring A}, Telschow F, Schwartzman A, Nichols TE. Spatial confidence sets for raw effect size images. \textmyfont{NeuroImage.} 2019. 

\textbf{Bowring A}, Maumet C, Nichols TE. Exploring the impact of analysis software on task fMRI results. \textmyfont{Human Brain Mapping.} 2019.\\
\textbf{Altmetric Attention Score ranking in the top 5\% of all research outputs scored by Altmetric.}

Botvinik-Nezer R, Holzmeister  F, \dots, \textbf{Bowring A}, \dots, Schonberg T. Variability in the analysis of a single neuroimaging dataset by many teams. \textmyfont{Nature.} 2020. 

Maumet C, Auer T, \textbf{Bowring A}, \dots, Nichols TE. Sharing brain mapping statistical results with the neuroimaging data model. \textmyfont{Nature Scientific Data.} 2016.

Pauli R, \textbf{Bowring A}, Reynolds R, Chen G, Nichols TE, Maumet C. Exploring fMRI Results Space: 31 Variants of an fMRI Analysis in AFNI, FSL, and SPM. \textmyfont{Frontiers in Neuroinformatics.} 2016.

\leftskip=0em
\parindent 0em

\section{\faBarChart}{Selected Conference Presentations}
\leftskip=2em
\parindent=-2em

\hspace{-2em}(Poster Presentation, Upcoming) Identifying Sources of Software-dependent Differences in Task fMRI Analyses, \textmyfont{27th Annual Meeting of the Organization for Human Brain Mapping}, Online (due to Covid-19 pandemic), June 21-25 2021.\\
\textbf{One of <5\% of submissions selected by the Program Committee as an exceptional abstract. }

(Poster Presentation) Spatial Confidence Sets for Standardized Effect Size Images, \textmyfont{26th Annual Meeting of the Organization for Human Brain Mapping}, Online (due to Covid-19 pandemic), June 23-July 3 2020. 

(Twitter Presentation) \href{https://twitter.com/OHBMequinoX/status/1240926466302500864}{Confidence Sets for Cohen’s \textmyfont{d} Effect Size Images}, \textmyfont{Organization for Human Brain Mapping Equinox Twitter Conference}, Online, March 20 2020. 

(Oral \& Poster Presentation) \href{https://www.pathlms.com/ohbm/courses/8246/sections/12541/video_presentations/116000}{Same Data – Different Software – Different Results? Analytic Variability of Group fMRI Results.}, \textmyfont{24th Annual Meeting of the Organization for Human Brain Mapping}, Singapore, June 17-21 2018. \\
\textbf{Winner of a \$2,000 OHBM Merit Abstract Award.} \\
\textbf{One of <5\% of submissions selected by the Program Committee for an oral presentation. }

(Oral \& Poster Presentation) \href{https://www.pathlms.com/ohbm/courses/5158/sections/7816/video_presentations/78445}{Spatial Confidence Sets - Beyond Null Hypothesis Testing of Cluster Size}, \textmyfont{23rd Annual Meeting of the Organization for Human Brain Mapping}, Vancouver, Canada, June 25-29 2017. \\
\textbf{One of <5\% of submissions selected by the Program Committee for an oral presentation. }

(Poster Presentation) Impact of Analysis Software on Replication of fMRI Studies, \textmyfont{23rd Annual Meeting of the Organization for Human Brain Mapping}, Vancouver, Canada, June 25-29 2017.

(Oral Presentation) Towards reproducible brain imaging research, \textmyfont{WIN Annual Conference}, Coventry, January 24 2017.

(Poster Presentation) Confidence Sets - Going Beyond Voxel-level and Cluster-level Null Hypothesis Testing, \textmyfont{22nd Annual Meeting of the Organization for Human Brain Mapping}, Geneva, Switzerland, June 26-30 2016.

\leftskip=0em
\parindent=0em

\section{\faPencilSquareO}{Teaching Experience}

\uline{\large\bfseries{University of Oxford}  \hfill \large\bfseries{Oxford, UK}}
{\begin{addmargin}[2em]{0em}

    \textmyfont{Data Sherpa for Covid-19 Data Challenge \hfill Michaelmas 2020}
    \begin{itemize}[topsep=0pt,itemsep=0pt,partopsep=0pt, parsep=0pt] 
    \item Mentored three PhD students in analysing Covid-19 data obtained from Oxfordshire hospitals to investigate the impact of Covid-19 on diabetes and glycaemic control.
    \item Prepared the data for the students, carrying out data cleaning on electronic health records from thousands of hospital patients using Python.
    \end{itemize}
    \textmyfont{Statistics Demonstrator \hfill Michaelmas 2019, 2020}
    \begin{itemize}[topsep=0pt,itemsep=0pt,partopsep=0pt, parsep=0pt] 
    \item Led online tutorials for PhD students enrolled in the graduate Statistics course where I assisted in practical exercises on modern statistical methods and machine learning.
    \item Taught students the fundamentals of regularised regression and Gaussian predictive models for machine learning.
    \end{itemize}
    \textmyfont{Introduction to Python Course Leader \hfill Michaelmas 2020}
    \begin{itemize}[topsep=0pt,itemsep=0pt,partopsep=0pt, parsep=0pt] 
    \item Ran an introductory course on Python programming for a class of graduate students enrolled in the Centre for Doctoral Training PhD programme.
    \item Taught students about how to use the fundamental functions and packages needed for programming in Python, including tutorials on the numpy, pandas and scipy packages.
    \end{itemize}
\end{addmargin}}
\vspace{0.2cm}
\uline{\large\bfseries{University of Warwick}  \hfill \large\bfseries{Coventry, UK}}
{\begin{addmargin}[2em]{0em}
    \textmyfont{Statistics Tutor \hfill Spring 2017}
    \begin{itemize}[topsep=0pt,itemsep=0pt,partopsep=0pt, parsep=0pt] 
    \item Led tutorials for a group of 20 undergraduate Statistics students taking the ST111 and ST112: Probabiliy A \& B modules.
    \item Taught students about the foundational notions of mathematical probability and marked assignments for each module.
     \end{itemize}
\end{addmargin}}
\vspace{0.2cm}
\uline{\large\bfseries{Teach First Insight Programme}  \hfill \large\bfseries{London, UK}}
{\begin{addmargin}[2em]{0em}
    \textmyfont{Teaching Intern \hfill Summer 2015}
    \begin{itemize}[topsep=0pt,itemsep=0pt,partopsep=0pt, parsep=0pt] 
    \item Selected for a summer internship with Teach First to help tackle social inequality in the UK education system.
    \item Planned and delivered lessons to a class of 30+ students at Cranford Community College, Hounslow, interacting with students between the ages of 11 and 18.  
    \item Gave a presentation on my experiences to senior Teach First employees in Canary Wharf. 
     \end{itemize}
\end{addmargin}}

   
\section{\faPencil}{Further Training}

\work{University of Oxford \hfill Oxford, UK}{\uline{\textmyfont{fMRIB Graduate Programme} \hfill \textmyfont{Oct 2017 - Jul 2017}}}%
    {\begin{itemize}[topsep=0pt,itemsep=0pt,partopsep=0pt, parsep=0pt] 
    \item Learned about the scanning and analysis methodologies used for three different modalities of neuroimaging: functional MRI, structural MRI, and diffusion MRI.  
    \item Carried out practical examples using the methods I was taught within FSL. 
    \item Completed an exam at the end of each term. For the MRI Physics and MRI Analysis exams, I achieved scores of 91\% and 88\% respectively. 
    \end{itemize}
    }%

\work{Lake Como School of Advanced Studies \hfill Lake Como, Italy}{\uline{\textmyfont{Bocconi Summer School in Advanced Statistics and Probability} \hfill \textmyfont{Jul 2017}}}%
    {\begin{itemize}[topsep=0pt,itemsep=0pt,partopsep=0pt, parsep=0pt] 
    \item One of 30 international students selected to participate in the summer school. 
    \item Learned about how statistical causal learning models can provide insight into machine learning tasks such as domain adaptation, transfer learning, and semi-supervised learning. 
    \end{itemize}
    }%
     
\work{Cambridge/Oxford/Durham/Glasgow Universities \hfill UK}%
    {\uline{\textmyfont{Academy for PhD Training in Statistics (APTS)} \hfill \textmyfont{Dec 2016 - Aug 2017}}}%
    {\begin{itemize}[topsep=0pt,itemsep=0pt,partopsep=0pt, parsep=0pt] 
    \item Completed four residential weeks of training in statistics and probability during the first year of my PhD (one week at each university).
    \item Learned about various aspects of statistical modelling, statistical inference and statistical computing across a total of eight intensive course modules.
    \end{itemize}
    }%

\section{\faGears}{SKILLS}

\textbf{Technical:} Python, R, Matlab, Unix, Github. \\
\textbf{Neuroimaging Analysis Software:} AFNI, FSL, SPM.  \\
\textbf{Research:} Data Preprocessing, Data Analysis, Statistical Testing, Simulations, Analysis of large neuroimaging datasets: UK Biobank, Human Connectome Project. \\

\section{\faLaptop}{Covid-19 County Case Tracker}

From March 2020-January 2021 I developed and maintained the \href{https://covid19cct.shinyapps.io/covid19cct/}{\textit{Covid-19 County Case Tracker}} R Shinyapp. I used R programming to harvest publicly available national Covid-19 data and provide visualisations of the trends in Covid-19 cases at the county-level in England and Scotland. During the first UK lockdown the app received over 100+ hours of usage per day, garnering attention from radio and media outlets including \href{https://grm.digital/heroes-of-covid19/info/}{GRM Digital} and the \href{https://www.oxfordmail.co.uk/news/18358052.app-tracks-oxfordshire-covid-19-cases/}{Oxford Mail}.

\end{document}
