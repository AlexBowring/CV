\documentclass{my_cv}
\usepackage[skins]{tcolorbox}
\usepackage{lipsum}
\usepackage[normalem]{ulem}
\usepackage{scrextend}
\usepackage{enumitem}
\usepackage{fontspec}

\newcommand*{\myfont}{\fontfamily{lmr}\selectfont}
\DeclareTextFontCommand{\textmyfont}{\itshape\myfont}

%Setup hyperref package, and colours for links
\usepackage{hyperref}
\definecolor{linkcolour}{rgb}{0,0.2,0.6}
\hypersetup{colorlinks,breaklinks,urlcolor=linkcolour, linkcolor=linkcolour}

\begin{document}
\begin{multicols}{2}[
    \titletext{}%
    		{Alexander Bowring}
        {Population Health PhD, Mathematics BSc}%
        {(+44) 7944 240731}%
        {\href{mailto:alex40466@hotmail.com}{alex40466@hotmail.com}}%
        {\href{https://www.linkedin.com/in/alexander-bowring/}{alexander-bowring}}%
        {\href{https://github.com/AlexBowring/}{AlexBowring}}
]

\end{multicols}

\vspace{-1.7cm}
\begin{tcolorbox}[colback=white,colframe=black,width=\dimexpr\textwidth+12mm\relax,enlarge left by=-6mm]
\section{\faStickyNote}{Summary}
\textbf{Senior Mathematical Consultant} and \textbf{Technical Lead}, passionate about applying \textbf{statistics}, \textbf{machine learning}, and \textbf{optimisation} to deliver actionable solutions to ambitious technical challenges. I have led technical teams across a \textbf{portfolio of projects exceeding £1M} in revenue, receiving \textbf{national awards} for innovation and impact. Skilled in programming languages: \textbf{Python}, \textbf{R}, \textbf{SQL}; libraries and frameworks: \textbf{LightGBM}, \textbf{PyTorch}, \textbf{pandas}, \textbf{scikit-learn}, \textbf{tidyverse}; and platforms and tools: \textbf{Azure DevOps}, \textbf{Git}, \textbf{AWS}, and \textbf{Power BI}. These skills are grounded in a technical background in statistics and mathematics, with a \textbf{PhD in Population Health from Oxford University} and \textbf{First-Class BSc in Mathematics from Warwick University}.


  
\end{tcolorbox}

\section{\faBriefcase}{Professional Experience}

\work{Senior Mathematical Consultant (previously Mathematical Consultant) \hfill Oxford, UK}{\uline{\textmyfont{Smith Institute} \hfill \textmyfont{Nov 2021 - Present}}}%
    {\begin{itemize}[topsep=0pt,itemsep=0pt,partopsep=0pt, parsep=0pt] 
\item \textbf{Lead technical teams} and \textbf{oversee all technical aspects of projects} delivering advanced mathematical solutions to \textbf{large industry clients}, collaborating closely with diverse stakeholders and \textbf{disseminating findings regularly}.
  \item Lead the \textbf{\href{https://www.neso.energy/news/dynamic-reserve-setting-innovation-project-seeks-reduce-balancing-costs}{Dynamic Reserve Setting (DRS)}} project with the National Energy System Operator (NESO), developing \textbf{explainable AI models} that are \textbf{productionised} in the control room with a Power BI dashboard, \textbf{saving NESO gigawatts in reserve procurement and millions of pounds annually}.
  \item DRS project recognised with the \textbf{\href{https://www.theorsociety.com/ORS/ORS/About-OR/News/Presidents-Medal-award.aspx}{Operational Research Society Presidents Medal Award 2024}} for the best practical application of operational research.
\item Principal model developer for the \textbf{\href{https://ssen-innovation.co.uk/nia-projects/vfes/}{Vulnerability Future Energy Scenarios (VFES)}} project with \textbf{Scottish and Southern Electricity Networks (SSEN)}, creating an \textbf{explainable AI model} to analyse vulnerability drivers and provide enhanced \textbf{decision intelligence} for strategic optimisation of customer support.
  \item VFES project recognised with the \textbf{\href{https://www.ssen.co.uk/news-views/2024/SSENs-innovative-net-zero-programmes-win-big-at-the-2024-Utility-Week-awards/}{Utility Week Unlocking Data Award 2024}} and the \textbf{\href{https://www.dataiq.global/award-winner/2024-dataiq-awards-best-use-of-ai-for-the-public-good/}{DataIQ Award for Best Use of AI for the Public Good 2024}}.
  \item Led the implementation of \textbf{classification models using convolutional neural networks (CNNs)} for a \textbf{defence and security project}.
    \end{itemize}
    }%

\work{Early Career Research Fellow in Neuroimaging Statistics \hfill Oxford, UK}{\uline{\textmyfont{Nuffield Department of Population Health, University of Oxford} \hfill \textmyfont{Nov 2019 - Nov 2021}}}%
{\begin{itemize}[topsep=0pt,itemsep=0pt,partopsep=0pt, parsep=0pt]    
    \item Led research on variability in MRI results caused by differences in analytic workflows, using \textbf{Python}, \textbf{Matlab}, and \textbf{Unix} to run parallel analyses on a computing cluster.
    \item Developed \textbf{statistical methods} to assess the similarity of spatial maps and created \textbf{Python notebooks} for analysis and visualization.
    \item Organized and led a weekly \textbf{Neuroimaging Statistics Oxford reading group} featuring international experts.
    \item Taught PhD students \textbf{statistical methods}, \textbf{programming skills}, and aspects of \textbf{independent research}.
\end{itemize}
}%

\work{Research Assistant in Neuroimaging Statistics \hfill Coventry, UK}{\uline{\textmyfont{Warwick Manufacturing Group, University of Warwick} \hfill \textmyfont{Nov 2015 - Oct 2016}}}%
{\begin{itemize}[topsep=0pt,itemsep=0pt,partopsep=0pt, parsep=0pt] 
    \item Developed standard practices for \textbf{data-sharing} and \textbf{meta-analysis} in neuroimaging, improving reproducibility and collaboration.
    \item Performed extensive \textbf{fMRI analyses} using major neuroimaging software and maintained version control with \textbf{GitHub}, sharing results via online repositories.
    \item Mentored a visiting \textbf{PhD student} on a short-term research project that was subsequently published.
\end{itemize}
}%
     
\work{Undergraduate Research Intern \hfill Coventry, UK}%
{\uline{\textmyfont{Undergraduate Research Support Scheme, University of Warwick} \hfill \textmyfont{Jun 2015 - Oct 2015}}}%
{\begin{itemize}[topsep=0pt,itemsep=0pt,partopsep=0pt, parsep=0pt] 
    \item Assisted in developing software to \textbf{visualise fMRI results} and tested research prototypes through \textbf{data analysis}.
    \item Gained hands-on experience in \textbf{collaborative research} and effective \textbf{communication} with supervisors and team members.
\end{itemize}
}%    

\section{\faGraduationCap}{Education}


\work{Doctor of Philosophy in Population Health \hfill Oxford, UK}{\uline{\textmyfont{Nuffield Department of Population Health, University of Oxford} \hfill \textmyfont{Oct 2016 - Nov 2019}}}%
{\begin{addmargin}[2em]{1em}
    Thesis: \href{https://doi.org/10.31237/osf.io/mj7qa}{On the Reproducibility and Interpretability of Group-Level Task-fMRI Results} 
    \begin{itemize}[topsep=0pt,itemsep=0pt,partopsep=0pt, parsep=0pt] 
        \item Developed a novel \textbf{Confidence Sets method} for group-level inference on task-fMRI effect-size maps, resulting in two publications in \textit{NeuroImage}.
        \item Evaluated reproducibility of task-fMRI results across software packages (AFNI, FSL, SPM), leading to a publication in \textit{Human Brain Mapping} ranked in the top 5\% of research outputs by Altmetric.
    \end{itemize}
\end{addmargin}}

\work{Bachelor of Science in Mathematics \hfill University of Warwick, UK}{\uline{\textmyfont{University of Warwick} \hfill \textmyfont{Sep 2012 - Jul 2015}}}%
{\begin{addmargin}[2em]{1em}
    Graduated with \textbf{First Class Honours}, achieving >80\% in modules including Algebra II, Analysis III, Galois Theory, Programming for Scientists, and Mathematics by Computer.
\end{addmargin}}

\section{\faPencil}{Further Information}
{\begin{addmargin}[2em]{1em}
\textbf{Publications:} Several publications in high-impact journals including \textit{Nature}, \textit{NeuroImage}, and \textit{Human Brain Mapping}. Full list available at \href{https://scholar.google.com/citations?user=h4aoOucAAAAJ}{Google Scholar}.

\hspace{-1.5em}\textbf{Technical Skills:} 
\begin{itemize}[topsep=0pt,itemsep=0pt,partopsep=0pt, parsep=0pt]
    \item \textbf{Analytical:} Frequentist statistics, machine learning, gradient boosting, linear optimisation, data visualisation, feature engineering, explainable AI, neural networks
    \item \textbf{Programming Languages:} Python, R, SQL, Bash, RDF, Cypher
    \item \textbf{Libraries / Frameworks:} LightGBM, XGBoost, optuna, PyTorch, pandas, polars, scikit-learn, numpy, matplotlib, tidyverse
    \item \textbf{Tools / Platforms:} Power BI, Git, Azure DevOps, AWS, Neo4j, Microsoft Excel
    \item \textbf{Development Environments:} PyCharm, RStudio
\end{itemize}
\end{addmargin}}
    
\section{\faLaptop}{Covid-19 County Case Tracker}
From March 2020-January 2021 I developed and maintained the \href{https://covid19cct.shinyapps.io/covid19cct/}{\textit{Covid-19 County Case Tracker}} R Shinyapp. I used R programming to harvest publicly available national Covid-19 data and provide visualisations of the trends in Covid-19 cases at the county-level in England and Scotland. During the first UK lockdown the app received over 100+ hours of usage per day, garnering attention from radio and media outlets including the \href{https://www.oxfordmail.co.uk/news/18358052.app-tracks-oxfordshire-covid-19-cases/}{Oxford Mail}.

%
%\section{\faBook}{Publications}
%\leftskip=2em
%\parindent=-2em
%
%\hspace{-2em}\textbf{Bowring A}, Telschow F, Schwartzman A, Nichols TE. Confidence Sets for Cohen's \textmyfont{d} effect size images. \textmyfont{NeuroImage.} 2021. 
%
%\textbf{Bowring A}, Telschow F, Schwartzman A, Nichols TE. Spatial confidence sets for raw effect size images. \textmyfont{NeuroImage.} 2019. 
%
%\textbf{Bowring A}, Maumet C, Nichols TE. Exploring the impact of analysis software on task fMRI results. \textmyfont{Human Brain Mapping.} 2019.\\
%\textbf{Altmetric Attention Score ranking in the top 5\% of all research outputs scored by Altmetric.}
%
%Botvinik-Nezer R, Holzmeister  F, \dots, \textbf{Bowring A}, \dots, Schonberg T. Variability in the analysis of a single neuroimaging dataset by many teams. \textmyfont{Nature.} 2020. 
%
%Maumet C, Auer T, \textbf{Bowring A}, \dots, Nichols TE. Sharing brain mapping statistical results with the neuroimaging data model. \textmyfont{Nature Scientific Data.} 2016.
%
%Pauli R, \textbf{Bowring A}, Reynolds R, Chen G, Nichols TE, Maumet C. Exploring fMRI Results Space: 31 Variants of an fMRI Analysis in AFNI, FSL, and SPM. \textmyfont{Frontiers in Neuroinformatics.} 2016.
%
%\leftskip=0em
%\parindent 0em
%
%\section{\faBarChart}{Selected Conference Presentations}
%\leftskip=2em
%\parindent=-2em
%
%\hspace{-2em}(Poster Presentation, Upcoming) Identifying Sources of Software-dependent Differences in Task fMRI Analyses, \textmyfont{27th Annual Meeting of the Organization for Human Brain Mapping}, Online (due to Covid-19 pandemic), June 21-25 2021.\\
%\textbf{One of <5\% of submissions selected by the Program Committee as an exceptional abstract. }
%
%(Poster Presentation) Spatial Confidence Sets for Standardized Effect Size Images, \textmyfont{26th Annual Meeting of the Organization for Human Brain Mapping}, Online (due to Covid-19 pandemic), June 23-July 3 2020. 
%
%(Twitter Presentation) \href{https://twitter.com/OHBMequinoX/status/1240926466302500864}{Confidence Sets for Cohen’s \textmyfont{d} Effect Size Images}, \textmyfont{Organization for Human Brain Mapping Equinox Twitter Conference}, Online, March 20 2020. 
%
%(Oral \& Poster Presentation) \href{https://www.pathlms.com/ohbm/courses/8246/sections/12541/video_presentations/116000}{Same Data – Different Software – Different Results? Analytic Variability of Group fMRI Results.}, \textmyfont{24th Annual Meeting of the Organization for Human Brain Mapping}, Singapore, June 17-21 2018. \\
%\textbf{Winner of a \$2,000 OHBM Merit Abstract Award.} \\
%\textbf{One of <5\% of submissions selected by the Program Committee for an oral presentation. }
%
%(Oral \& Poster Presentation) \href{https://www.pathlms.com/ohbm/courses/5158/sections/7816/video_presentations/78445}{Spatial Confidence Sets - Beyond Null Hypothesis Testing of Cluster Size}, \textmyfont{23rd Annual Meeting of the Organization for Human Brain Mapping}, Vancouver, Canada, June 25-29 2017. \\
%\textbf{One of <5\% of submissions selected by the Program Committee for an oral presentation. }
%
%(Poster Presentation) Impact of Analysis Software on Replication of fMRI Studies, \textmyfont{23rd Annual Meeting of the Organization for Human Brain Mapping}, Vancouver, Canada, June 25-29 2017.
%
%(Oral Presentation) Towards reproducible brain imaging research, \textmyfont{WIN Annual Conference}, Coventry, January 24 2017.
%
%(Poster Presentation) Confidence Sets - Going Beyond Voxel-level and Cluster-level Null Hypothesis Testing, \textmyfont{22nd Annual Meeting of the Organization for Human Brain Mapping}, Geneva, Switzerland, June 26-30 2016.
%
%\leftskip=0em
%\parindent=0em
%
%\section{\faPencilSquareO}{Teaching Experience}
%
%\uline{\large\bfseries{University of Oxford}  \hfill \large\bfseries{Oxford, UK}}
%{\begin{addmargin}[2em]{0em}
%
%    \textmyfont{Data Sherpa for Covid-19 Data Challenge \hfill Michaelmas 2020}
%    \begin{itemize}[topsep=0pt,itemsep=0pt,partopsep=0pt, parsep=0pt] 
%    \item Mentored three PhD students in analysing Covid-19 data obtained from Oxfordshire hospitals to investigate the impact of Covid-19 on diabetes and glycaemic control.
%    \item Prepared the data for the students, carrying out data cleaning on electronic health records from thousands of hospital patients using Python.
%    \end{itemize}
%    \textmyfont{Statistics Demonstrator \hfill Michaelmas 2019, 2020}
%    \begin{itemize}[topsep=0pt,itemsep=0pt,partopsep=0pt, parsep=0pt] 
%    \item Led online tutorials for PhD students enrolled in the graduate Statistics course where I assisted in practical exercises on modern statistical methods and machine learning.
%    \item Taught students the fundamentals of regularised regression and Gaussian predictive models for machine learning.
%    \end{itemize}
%    \textmyfont{Introduction to Python Course Leader \hfill Michaelmas 2020}
%    \begin{itemize}[topsep=0pt,itemsep=0pt,partopsep=0pt, parsep=0pt] 
%    \item Ran an introductory course on Python programming for a class of graduate students enrolled in the Centre for Doctoral Training PhD programme.
%    \item Taught students about how to use the fundamental functions and packages needed for programming in Python, including tutorials on the numpy, pandas and scipy packages.
%    \end{itemize}
%\end{addmargin}}
%\vspace{0.2cm}
%\uline{\large\bfseries{University of Warwick}  \hfill \large\bfseries{Coventry, UK}}
%{\begin{addmargin}[2em]{0em}
%    \textmyfont{Statistics Tutor \hfill Spring 2017}
%    \begin{itemize}[topsep=0pt,itemsep=0pt,partopsep=0pt, parsep=0pt] 
%    \item Led tutorials for a group of 20 undergraduate Statistics students taking the ST111 and ST112: Probabiliy A \& B modules.
%    \item Taught students about the foundational notions of mathematical probability and marked assignments for each module.
%     \end{itemize}
%\end{addmargin}}
%\vspace{0.2cm}
%\uline{\large\bfseries{Teach First Insight Programme}  \hfill \large\bfseries{London, UK}}
%{\begin{addmargin}[2em]{0em}
%    \textmyfont{Teaching Intern \hfill Summer 2015}
%    \begin{itemize}[topsep=0pt,itemsep=0pt,partopsep=0pt, parsep=0pt] 
%    \item Selected for a summer internship with Teach First to help tackle social inequality in the UK education system.
%    \item Planned and delivered lessons to a class of 30+ students at Cranford Community College, Hounslow, interacting with students between the ages of 11 and 18.  
%    \item Gave a presentation on my experiences to senior Teach First employees in Canary Wharf. 
%     \end{itemize}
%\end{addmargin}}
%
%   
%\section{\faPencil}{Further Training}
%
%\work{University of Oxford \hfill Oxford, UK}{\uline{\textmyfont{fMRIB Graduate Programme} \hfill \textmyfont{Oct 2017 - Jul 2017}}}%
%    {\begin{itemize}[topsep=0pt,itemsep=0pt,partopsep=0pt, parsep=0pt] 
%    \item Learned about the scanning and analysis methodologies used for three different modalities of neuroimaging: functional MRI, structural MRI, and diffusion MRI.  
%    \item Carried out practical examples using the methods I was taught within FSL. 
%    \item Completed an exam at the end of each term. For the MRI Physics and MRI Analysis exams, I achieved scores of 91\% and 88\% respectively. 
%    \end{itemize}
%    }%
%
%\work{Lake Como School of Advanced Studies \hfill Lake Como, Italy}{\uline{\textmyfont{Bocconi Summer School in Advanced Statistics and Probability} \hfill \textmyfont{Jul 2017}}}%
%    {\begin{itemize}[topsep=0pt,itemsep=0pt,partopsep=0pt, parsep=0pt] 
%    \item One of 30 international students selected to participate in the summer school. 
%    \item Learned about how statistical causal learning models can provide insight into machine learning tasks such as domain adaptation, transfer learning, and semi-supervised learning. 
%    \end{itemize}
%    }%
%     
%\work{Cambridge/Oxford/Durham/Glasgow Universities \hfill UK}%
%    {\uline{\textmyfont{Academy for PhD Training in Statistics (APTS)} \hfill \textmyfont{Dec 2016 - Aug 2017}}}%
%    {\begin{itemize}[topsep=0pt,itemsep=0pt,partopsep=0pt, parsep=0pt] 
%    \item Completed four residential weeks of training in statistics and probability during the first year of my PhD (one week at each university).
%    \item Learned about various aspects of statistical modelling, statistical inference and statistical computing across a total of eight intensive course modules.
%    \end{itemize}
%    }%
%
%\section{\faGears}{Skills}
%
%\textbf{Technical:} Python, R, Matlab, Unix, Github. \\
%\textbf{Neuroimaging Analysis Software:} AFNI, FSL, SPM.  \\
%\textbf{Research:} Data Preprocessing, Data Analysis, Statistical Testing, Simulations, Analysis of large neuroimaging datasets: UK Biobank, Human Connectome Project. \\
%
%\section{\faLaptop}{Covid-19 County Case Tracker}
%
%From March 2020-January 2021 I developed and maintained the \href{https://covid19cct.shinyapps.io/covid19cct/}{\textit{Covid-19 County Case Tracker}} R Shinyapp. I used R programming to harvest publicly available national Covid-19 data and provide visualisations of the trends in Covid-19 cases at the county-level in England and Scotland. During the first UK lockdown the app received over 100+ hours of usage per day, garnering attention from radio and media outlets including \href{https://grm.digital/heroes-of-covid19/info/}{GRM Digital} and the \href{https://www.oxfordmail.co.uk/news/18358052.app-tracks-oxfordshire-covid-19-cases/}{Oxford Mail}.

\end{document}
