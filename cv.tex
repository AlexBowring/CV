\documentclass[a4paper,10pt]{article}

%A Few Useful Packages
\usepackage{marvosym}
\usepackage{fontspec} 					%for loading fonts
\usepackage{xunicode,xltxtra,url,parskip} 	%other packages for formatting
\RequirePackage{color,graphicx}
\usepackage[usenames,dvipsnames]{xcolor}
\usepackage[big]{layaureo} 				%better formatting of the A4 page
% an alternative to Layaureo can be ** \usepackage{fullpage} **
\usepackage{supertabular} 				%for Grades
\usepackage{titlesec}					%custom \section
\usepackage{longtable}

%Setup hyperref package, and colours for links
\usepackage{hyperref}
\definecolor{linkcolour}{rgb}{0,0.2,0.6}
\hypersetup{colorlinks,breaklinks,urlcolor=linkcolour, linkcolor=linkcolour}

%FONTS
\defaultfontfeatures{Mapping=tex-text}
%\setmainfont[SmallCapsFont = Fontin SmallCaps]{Fontin}
%%% modified for Karol Kozioł for ShareLaTeX use
\setmainfont[
SmallCapsFont = Fontin-SmallCaps.otf,
BoldFont = Fontin-Bold.otf,
ItalicFont = Fontin-Italic.otf
]
{Fontin.otf}
%%%

%CV Sections inspired by: 
%http://stefano.italians.nl/archives/26
\titleformat{\section}{\Large\scshape\raggedright}{}{0em}{}[\titlerule]
\titlespacing{\section}{0pt}{3pt}{3pt}

%Tweak a bit the top margin
\addtolength{\voffset}{-1.0cm}

	\addtolength{\oddsidemargin}{-.70in}
	\addtolength{\evensidemargin}{-.70in}
	\addtolength{\textwidth}{1.4in}

	\addtolength{\textheight}{1.4in}

%Italian hyphenation for the word: ''corporations''
\hyphenation{im-pre-se}

\usepackage{blindtext}
\usepackage{booktabs}
\usepackage{enumitem}

%-------------WATERMARK TEST [**not part of a CV**]---------------
\usepackage[absolute]{textpos}

\setlength{\TPHorizModule}{30mm}
\setlength{\TPVertModule}{\TPHorizModule}
\textblockorigin{2mm}{0.65\paperheight}
\setlength{\parindent}{0pt}

%--------------------BEGIN DOCUMENT----------------------
\begin{document}

%WATERMARK TEST [**not part of a CV**]---------------
%\font\wm=''Baskerville:color=787878'' at 8pt
%\font\wmweb=''Baskerville:color=FF1493'' at 8pt
%{\wm 
%	\begin{textblock}{1}(0,0)
%		\rotatebox{-90}{\parbox{500mm}{
%			Typeset by Alessandro Plasmati with \XeTeX\  \today\ for 
%			{\wmweb \href{http://www.aleplasmati.comuv.com}{aleplasmati.comuv.com}}
%		}
%	}
%	\end{textblock}
%}

\pagestyle{empty} % non-numbered pages

\font\fb=''[cmr10]'' %for use with \LaTeX command

%--------------------TITLE-------------
\par{\centering
		{\Huge Alexander \textsc{Bowring}
	}\bigskip\par}

\begin{center}
\begin{tabular}{rl}
	\textsc{address:} & Apt. 40 Trinity Court, 4 Between Towns Road, Oxford, OX4 3PP \\
	\textsc{phone:} & +447944 240731\\
    \textsc{email:}     & \href{mailto:alex.bowring@bdi.ox.ac.uk}{alex.bowring@bdi.ox.ac.uk}\\
    \\
    \textbf{Research Interests:} & Functional Magnetic Resonance Imaging, Analysis Pipelines, \\& Statistical Inference Methods, Open science.
\end{tabular}
\end{center}


%--------------------SECTIONS-----------------------------------
%Section: Personal Data
%\section{Personal Data}

%\begin{tabular}{rl}
%    \textsc{Place and Date of Birth:} & Someplace, Italy  | dd Month 1912 \\
%   \textsc{Address:}   & CV Inn 19, 20301, Milano, Italy \\
%    \textsc{Phone:}     & +39 123 456789\\
%    \textsc{email:}     & \href{mailto:alessandro.plasmati@gmail.com}{alessandro.plasmati@gmail.com}
%    \end{tabular}

%Section: Education
\section{Education}
\begin{tabular}{rl}	
\textsc{Oct} 2017 - Present & DPhil. in \textsc{Population Health}, \textbf{The University of Oxford}, Oxford\\
& \textbf{Thesis:} \textit{A Comparison of Neuroimaging Software and a Contour Inference Method} \\& \textit{for Analysis of Task-fMRI Data.}\\
& \textbf{Supervisors:} \textit{Professor Thomas Nichols, PhD. \& Professor Stephen Smith, PhD.} \\&\\ 
\textsc{Oct} 2016 - \textsc{Oct} 2017 & PhD. in \textsc{Statistics}, \textbf{The University of Warwick}, Coventry\\
& \textbf{Supervisors:} \textit{Professor Thomas Nichols, PhD. \& Professor Armin Schwartzman, PhD.} \\&\\
\textsc{Sep} 2012 - \textsc{Jul} 2015 & BSc. in \textsc{Mathematics}, \textbf{The University of Warwick}, Coventry\\
& \textbf{First Class Honours.} \\
& Selected modules: \textit{Galois Theory 88\%, Analysis III 90\%, Algebra II 86\%, Algebra I 80\%,}\\
&\textit{Functional Analysis II 78\%, Programming for Scientists 89\%, Mathematics by Computer 92\%.}\\&\\
\textsc{Sep} 2009 - \textsc{Jul} 2012 & GCE A Levels, \textbf{The College of Richard Collyer}, Horsham\\
& \textbf{A Level results:} \textit{Further Mathematics A*, Mathematics A*, Economics A, Physics C.}\\&\\
\end{tabular}

%Section: Scholarships and additional info
\section{Publications}
(Preprint) \textbf{Bowring A}, Telschow F, Schwartzman A, Nichols TE. Spatial Confidence Sets for Raw Effect Size Images. \textit{BioRxiv} 2019. 

\textbf{Bowring A}, Maumet C, Nichols TE. Exploring the impact of analysis software on task fMRI results. \textit{Human Brain Mapping.} 2019. 

Maumet C, Auer T, \textbf{Bowring A}, \dots, Nichols TE. Sharing brain mapping statistical results with the neuroimaging data model. \textit{Nature Scientific Data} 2016.

Pauli R, \textbf{Bowring A}, Reynolds R, Chen G, Nichols TE, Maumet C. Exploring fMRI Results Space: 31 Variants of an fMRI Analysis in AFNI, FSL, and SPM. \textit{Frontiers in Neuroinformatics.} 2016.\\

\section{Selected Conference Presentations}
(Oral \& Poster Presentation) \textbf{Same Data – Different Software – Different Results? Analytic Variability of Group fMRI Results.}, \textit{24th Annual Meeting of the Organization for Human Brain Mapping}, Singapore, June 17-21 2018. 
\vspace{-2mm}
\begin{itemize}
\item Winner of a \$2,000 OHBM Merit Abstract Award.
\vspace{-2mm}
\item Less than 5\% of submitted abstracts chosen for an oral presentation. 
\end{itemize}

(Oral \& Poster Presentation) \textbf{Spatial Confidence Sets - Beyond Null Hypothesis Testing of Cluster Size}, \textit{23rd Annual Meeting of the Organization for Human Brain Mapping}, Vancouver, Canada, June 25-29 2017.
\vspace{-2mm}
\begin{itemize}
\item Less than 5\% of submitted abstracts chosen for an oral presentation. 
\end{itemize}

(Poster Presentation) \textbf{Impact of Analysis Software on Replication of fMRI Studies}, \textit{23rd Annual Meeting of the Organization for Human Brain Mapping}, Vancouver, Canada, June 25-29 2017.

(Oral Presentation) \textbf{Towards reproducible brain imaging research}, \textit{WIN Annual Conference}, Coventry, January 24 2017.

(Poster Presentation) \textbf{Confidence Sets - Going Beyond Voxel-level and Cluster-level Null Hypothesis Testing.} \textit{22nd Annual Meeting of the Organization for Human Brain Mapping}, Geneva, Switzerland, June 26-30 2016. 

\newpage 
%Section: Employment
\section{Research Experience}
\begin{longtable}{rl}	
 \textsc{Nov} 2015 - \textsc{Oct} 2016
& \large{\textbf{Research Assistant in Neuroimaging Statistics}}\\
& \textsc{Warwick Manufacturing Group}\\
& \textit{The Instititute of Digital Healthcare, The University of Warwick, Coventry} \\
&\begin{minipage}[t]{0.8\textwidth}
 \begin{itemize}[leftmargin=*]
 %\setlength{\itemindent}{-.5cm}
 \item Worked alongside Prof. Thomas Nichols and Dr. Camille Maumet to develop standard practices for data sharing and meta-analysis in neuroimaging.
 \item Extensively analysed neuroimaging data, gaining significant experience of the three main neuroimaging software packages: \textit{SPM}, \textit{FSL} and \textit{AFNI}. 
 \item Made analyses publicly available and provided documentation on my research efforts, becoming familiar with the Github version control system and making use of online data repositories such as Neurovault. 
 \item Mentored a PhD student who visited our lab for three months.  Our research during this period was published after peer-review in the journal \textit{Frontiers in Neuroinformatics}, for which I am second author. 
 \item Assisted Dr. Maumet on the \textit{Neuroimaging Data Model} (NIDM), a project dedicated to the development of a standard to share neuroimaging results.
 \item As part of the NIDM project, proof-read and edited specification documents and tested research prototypes, reporting back on any issues. 
 \item Analyzed data and helped edit a journal article for NIDM, now published in \textit{Nature Scientific Data}, for which I am a coauthor. 
 \item Participated in weekly conference calls with international collaborators, discussing progress and ideas to further develop the NIDM project.
 \item Conducted my own research into developing a spatial inference method for analysing neuroimaging data under the supervision of Prof. Nichols. 
 \item Created and ran computer simulations to test this method in MATLAB. 
 \item Wrote an abstract describing the method, and gave a poster presentation on this research at the \textit{Organization for Human Brain Mapping} conference in Geneva, Switzerland.  
 \item Communicated daily with Prof. Nichols and Dr. Maumet, and participated in weekly meetings with other post-doctoral researchers and PhD students working in the field at the university. 
  \end{itemize}
 \end{minipage} \\&\\
 
\textsc{Jun} 2015 - \textsc{Oct} 2015
& \large{\textbf{Research Intern}}\\
& \textsc{Undergraduate Research Support Scheme}\\
& \textit{The University of Warwick, Coventry} \\
&\begin{minipage}[t]{0.8\textwidth}
 \begin{itemize}[leftmargin=*]
 %\setlength{\itemindent}{-.5cm}
 \item Worked on the project \textit{Visualising the brain - Developing viewers of standardised fMRI results} under the supervision of Prof. Thomas Nichols.
 \item Analysed neuroimaging data in the neuroimaging software package \textit{SPM}. I used this data to test a research prototype developed by Dr Camille Maumet and made my analyses publicly available online.  
 \item Communicated daily with Prof. Nichols and participated in weekly department meetings with other post-doctoral researchers and PhD students.
\end{itemize}
\end{minipage}\\&\\
\end{longtable}

\section{Teaching Experience}


\begin{longtable}{rl}
\textsc{Jan} 2017 - \textsc{Mar} 2017
& \large{\textbf{Statistics Tutor}}\\
& \textsc{The University of Warwick}\\
& \textit{Coventry} \\
&\begin{minipage}[t]{0.8\textwidth}
 \begin{itemize}[leftmargin=*]
 %\setlength{\itemindent}{-.5cm}
 \item Led tutorials for a group of 20 students taking the ST111 and ST112: Probability A \& B undergraduate statistics modules.
 \item Taught students about the foundational notions of mathematical probability and marked assignments for each module.
\end{itemize}
\end{minipage}\\&\\

\ \ \ \ \ \ \ \ \ \ \ \ \ \ \ \ \ \textsc{Sep} 2015
& \large{\textbf{Teaching Intern}}\\
& \textsc{Teach First Insight Programme}\\
& \textit{London} \\
&\begin{minipage}[t]{0.8\textwidth}
 \begin{itemize}[leftmargin=*]
 %\setlength{\itemindent}{-.5cm}
 \item One of a group of five interns that worked at Cranford Community College, Hounslow.
 \item Assisted with lessons across a range of subject areas, interacting with students between the ages of 11 and 18.   
 \item Planned and delivered a lesson on prime factorisation to a group of thirty-two Year 8 students.
 \item Gave a presentation on my experiences at the school to Teach First employees in Canary Wharf. \end{itemize}
\end{minipage}\\&\\

\end{longtable}

%Section: Work Experience at the top
\section{Further Training}
\begin{tabular}{rl}
\textsc{Oct} 2017 - \textsc{Jun} 2018
& \large{\textbf{fMRIB Graduate Programme}}\\
& \textsc{The University of Oxford}\\
& \textit{Oxford} \\
&\begin{minipage}[t]{0.8\textwidth}
 \begin{itemize}[leftmargin=*]
 %\setlength{\itemindent}{-.5cm}
 \item Learnt about the scanning and analysis methodologies used for three different modalities of neuroimaging: functional MRI, structural MRI, and diffusion MRI. 
 \item Carried out practical examples using the methods I was taught within FSL. 
 \item Completed an exam at the end of each term. For the MRI Physics and MRI Analysis exams, I achieved scores of 91\% and 88\% respectively. 
 \end{itemize}
\end{minipage}\\&\\

\ \ \ \ \ \ \ \ \ \ \ \ \ \ \ \ \ \textsc{Jul} 2017
& \large{\textbf{Bocconi Summer School in Advanced Statistics and Probability}}\\
& \textsc{Lake Como School of Advanced Studies}\\
& \textit{Lake Como, Italy} \\
&\begin{minipage}[t]{0.8\textwidth}
 \begin{itemize}[leftmargin=*]
 %\setlength{\itemindent}{-.5cm}
 \item One of 30 international students selected to participate in the summer school.   
 \item Learnt about how statistical causal learning models can provide insight into machine learning tasks such as domain adaptation, transfer learning, and semi-supervised learning. 
\end{itemize}
\end{minipage}\\&\\

\textsc{Dec} 2016 - \textsc{Aug} 2017
& \large{\textbf{Academy for PhD Training in Statistics (APTS)}}\\
& \textsc{Cambridge University, Oxford University, Durham University, Glasgow University}\\
& \textit{UK} \\
&\begin{minipage}[t]{0.8\textwidth}
 \begin{itemize}[leftmargin=*]
 %\setlength{\itemindent}{-.5cm}
 \item Completed four residential weeks of training in statistics and probability during the first year of my PhD (one week at each university).
 \item Learnt about various aspects of statistical modelling, statistical inference and statistical computing across a total of eight intensive course modules. 
\end{itemize}
\end{minipage}\\&\\

\end{tabular}

%Section: Languages

\section{IT Skills}
\begin{minipage}[t]{0.9\textwidth}
 \begin{itemize}
 %\setlength{\itemindent}{-.5cm}
 \item \textbf{Programming Languages:} MATLAB, Python, bash, JAVA.
 \item \textbf{Neuroimaging Software:} AFNI, FSL, SPM. 
 \item \textbf{Version Control:} Git (Github). 
 \end{itemize}
\end{minipage}

\section{References}
\begin{enumerate}
\item \textbf{Professor Thomas Nichols} \\
\textit{Big Data Institute, Li Ka Shing Centre for Health Information and Discovery, Nuffield Department of Population
Health, University of Oxford, Oxford, UK.} \\
\textbf{Email:} thomas.nichols@bdi.ox.ac.uk\\
	
\item \textbf{Doctor Camille Maumet} \\
\textit{Inria, Univ Rennes, CNRS, Inserm, IRISA UMR 6074, Empenn ERL U 1228, Rennes, France.} \\
\textbf{Email:} camille.maumet@inria.fr \\
\end{enumerate}


%\newpage
%\hypertarget{gmat}{\textsc{Gmat}\setmainfont{LMRoman10 Regular}\textregistered\setmainfont[SmallCapsFont=Fontin-SmallCaps]{Fontin-Regular}}

%\XeTeXpdffile ''GMAT.pdf'' page 1 scaled 800

\end{document}
